% Copyright 2004 by Till Tantau <tantau@users.sourceforge.net>.
%
% In principle, this file can be redistributed and/or modified under
% the terms of the GNU Public License, version 2.
%
% However, this file is supposed to be a template to be modified
% for your own needs. For this reason, if you use this file as a
% template and not specifically distribute it as part of a another
% package/program, I grant the extra permission to freely copy and
% modify this file as you see fit and even to delete this copyright
% notice. 



\documentclass{beamer}
\usepackage{amsfonts}
\usepackage{amsthm}
\usepackage{bm}
\usepackage{amsmath}
\usepackage{braket}
\usepackage{tikz}
\usepackage{amssymb}

\usepackage[T1]{fontenc}
\usepackage{tgbonum}




% There are many different themes available for Beamer. A comprehensive
% list with examples is given here:
% http://deic.uab.es/~iblanes/beamer_gallery/index_by_theme.html
% You can uncomment the themes below if you would like to use a different
% one:
%\usetheme{AnnArbor}
%\usetheme{Antibes}
%\usetheme{Bergen}
%\usetheme{Berkeley}
%\usetheme{Berlin}
%\usetheme{Boadilla}
%\usetheme{boxes}
%\usetheme{CambridgeUS}
%\usetheme{Copenhagen}
%\usetheme{Darmstadt}
%\usetheme{default}
%\usetheme{Frankfurt}
%\usetheme{Goettingen}
%\usetheme{Hannover}
%\usetheme{Ilmenau}
%\usetheme{JuanLesPins}
%\usetheme{Luebeck}
\usetheme{metropolis}
%\usetheme{Malmoe}
%\usetheme{Marburg}
%\usetheme{Montpellier}
%\usetheme{PaloAlto}
%\usetheme{Pittsburgh}
%\usetheme{Rochester}
%\usetheme{Singapore}
%\usetheme{Szeged}
%\usetheme{Warsaw}

\title{Improved Non-Interactive Zero Knowledge with Applications to Post-Quantum Signatures\\ by  J. Katz, V. Kolesnikov, X. Wang (KKW)}

% A subtitle is optional and this may be deleted
%\subtitle{Optional Subtitle}

\author{Khai Hanh Tang}
% - Give the names in the same order as the appear in the paper.
% - Use the \inst{?} command only if the authors have different
%   affiliation.

% - Use the \inst command only if there are several affiliations.
% - Keep it simple, no one is interested in your street address.

\date{31 May, 2022}
% - Either use conference name or its abbreviation.
% - Not really informative to the audience, more for people (including
%   yourself) who are reading the slides online

%\subject{Theoretical Computer Science}
% This is only inserted into the PDF information catalog. Can be left
% out. 

% If you have a file called "university-logo-filename.xxx", where xxx
% is a graphic format that can be processed by latex or pdflatex,
% resp., then you can add a logo as follows:

% \pgfdeclareimage[height=0.5cm]{university-logo}{university-logo-filename}
% \logo{\pgfuseimage{university-logo}}

% Delete this, if you do not want the table of contents to pop up at
% the beginning of each subsection:
\AtBeginSubsection[]
{
	\begin{frame}<beamer>{Outline}
		\tableofcontents[currentsection,currentsubsection]
	\end{frame}
}

% Let's get started

\newcommand{\Coll}[0]{\operatorname{Coll}}

\usepackage[utf8]{inputenc}
\usepackage[T1]{fontenc}

\newcommand{\uniformly}{\stackrel{\$}{\leftarrow}}
\newcommand{\nullity}{\mathsf{nullity}}

\newtheorem{construction}{Construction}
\newtheorem{claim}{Claim}
\newtheorem{algorithm}{Algorithm}

\begin{document}
	
	\begin{frame}
		\titlepage
	\end{frame}
	
	\begin{frame}{Outline}
		\tableofcontents
		
		% You might wish to add the option [pausesections]
	\end{frame}
	
	\section{Introduction}
	\begin{frame}{Introduction}
		\begin{itemize}
			\item \cite{KatzK018} showed a novel way to instantiate ``MPC-in-the-head'' approach \cite{IshaiKOS07} for circuits of around $300$ to $10^5$ gates.\pause
			
			\item The result is designed in the ``preprocessing model'' where preprocessings can be conducted before witnesses are known.\pause
			
			\item Offer 3.2-time shorter proof size compared to Chase et al. \cite{ChaseDGORRSZ17} (Picnic 1).\pause
			
			\item The resulting construction becomes Picnic 2 with further improvements \cite{KalesZ20} leading to Picnic 3.
		\end{itemize}
	\end{frame}
	
	\section{An MPC Protocol}
	\begin{frame}{Preliminaries}
		Let $C$ be a Boolean circuit with only AND ($\cdot$), XOR ($\oplus$) and NOT gates. \pause
		
		Define $\vert C \vert$ to be the number of AND gates of $C$.\pause
		
		Let $z_\alpha$ be the value at wire $\alpha$ of the circuit. In particular,\pause
		\begin{itemize}
			\item $z_\gamma = z_\alpha \oplus z_\beta$ for a XOR gate with output wire $\gamma$ and input wires $\alpha$ and $\beta$.\pause
			\item $z_\gamma = \overline{z}_\alpha$ for a NOT gate with output wire $\gamma$ and input wire $\alpha$.\pause
			\item $z_\gamma = z_\alpha \cdot z_\beta$ for an AND gate with output wire $\gamma$ and input wires $\alpha$ and $\beta$.
		\end{itemize}
	\end{frame}

	\begin{frame}{Masking}
		Mask each wire of the circuit as follows:\pause
		\begin{itemize}
			\item \textbf{Input wire:} Sample $\lambda_\alpha$ and compute $\hat{z}_\alpha := z_\alpha \oplus \lambda_\alpha$.\pause
			\item \textbf{Output of XOR gate:} Let $\lambda_\gamma = \lambda_\alpha \oplus \lambda_\beta$. Compute $\hat{z}_\gamma = z_\gamma \oplus \lambda_\gamma = (z_\alpha \oplus \lambda_\alpha) \oplus (z_\beta \oplus \lambda_\beta) =\hat{z}_\alpha \oplus \hat{z}_\beta$.\pause
			\item \textbf{Output of AND gate:} For $z_\gamma = z_\alpha \cdot z_\beta$, 
			
			sample $\lambda_\gamma$ independently and compute $\hat{z}_\gamma := z_\gamma \oplus \lambda_\gamma$.
		\end{itemize}
	\end{frame}

	\begin{frame}{Multiparty Computation}
		Run an MPC protocol ``in-the-head'' with $n$ parties $S_1, \dots, S_n$:\pause
		\begin{itemize}
			\item For each mask $\lambda_\alpha$, define $[\lambda_\alpha] = (\lambda_\alpha^{(1)}, \dots, \lambda_\alpha^{(n)})$ satisfying \pause
			\begin{equation*}
				\lambda_\alpha = \lambda_\alpha^{(1)} \oplus \dots\oplus \lambda_\alpha^{(n)}.\pause
			\end{equation*}
			\item Compute $\left(\hat{z}_\alpha\right)$ following circuit $C$ by using $\left(z_\alpha\right)$ and $\left([\lambda_\alpha]\right)$.\pause
		\end{itemize}
		\textbf{Challenge.} Compute $\hat{z}_\gamma$ at each output wire of each AND gate, since $\hat{z}_\gamma$ is independent of $\hat{z}_\alpha$ and $\hat{z}_\beta$.
	\end{frame}

	\begin{frame}{Computing $\hat{z}_\gamma$ for Output Wires of AND Gates}
		Additionally compute $\lambda_{\alpha,\beta} := \lambda_\alpha \cdot \lambda_\beta$ and distribute each share of $[\lambda_{\alpha,\beta}]$ to each party.\pause
		
		For each $i$ among $n$ parties, for each AND gate, compute $s^{(i)}_\gamma := \hat{z}_\alpha \cdot\lambda_\beta^{(i)}  \oplus \hat{z}_\beta\cdot\lambda_\alpha^{(i)}  \oplus \lambda_{\alpha,\beta}^{(i)}  \oplus \lambda_\gamma^{(i)} $.\pause
		
		Hence $[s_\gamma] = (s^{(1)}_\gamma, \dots, s^{(n)}_\gamma)$. \pause
		
		Each party broadcast $s^{(i)}$ to every other party. \pause
		
		Each party hence obtains $s$ and can compute $\hat{z}_\gamma := s_\gamma \oplus \hat{z}_\alpha \cdot \hat{z}_\beta$.
	\end{frame}

	\section{A $5$-Round ZKPoK}
	\begin{frame}{Technical Overview} 
		Prover runs $m$ instances of the MPC protocol.\pause
		\begin{itemize}
			\item \textbf{Round 1 (Prover).} For each $j \in [m]$, generate and commit all possible $(\lambda_\alpha)$ and $(\lambda_{\alpha,\beta})$.\pause
			\item \textbf{Round 2 (Verifier).} Send $c \in [m]$ to check the correct computation of instance $c$-th.\pause
			\item \textbf{Round 3 (Prover).} There are $2$ cases:\pause
			\begin{itemize}
				\item For all $j \neq c$, open all related commitments.\pause
				\item Simulate MPC w.r.t $c$-th instance, and commit the broadcast messages by $h' := H\left((s_\gamma^{(1)}), \dots, (s_\gamma^{(n)})\right)$.\pause
			\end{itemize}
			\item \textbf{Round 4 (Verifier).} Send $p \in [n]$.\pause
			\item \textbf{Round 5 (Prover).} Open commitments of all parties but $p$-th, and send $(s_\gamma^{(p)})$.\pause
			\item \textbf{Verification.} Verify opened commitments, and simulate $c$-th instance to compute broadcast messages as well as open $h'$.
		\end{itemize}
	\end{frame}

	\begin{frame}{Security}
		If commitment schemes and hash functions are secure, then \pause
		\begin{itemize}
			\item The $5$-round protocol is honest-verifier ZKPoK.\pause
			\item Soundness error is $\max\{\frac{1}{n}, \frac{1}{m}\}$.\pause
		\end{itemize}
		\textbf{Issue.} Can we make soundness smaller?
	\end{frame}

	\section{Transforming $5$ to $3$-Round ZKPoK}
	\begin{frame}{Technical Overview}
		Instead of verifying an instance of MPC protocol, do as follows:\pause
		\begin{itemize}
			\item Verify executions of a subset $\mathcal{C}$ of instances.\pause
			\item For each instance in $\mathcal{C}$, open commitments of all parties but $p$-th (challenge from verifier) and send $(s_\gamma^{(p)})$ to verifier for verification.\pause
		\end{itemize}
		
		\textbf{Achievement.} A controllable soundness error equal to 
		\begin{equation*}
			\epsilon(m, n, \tau) = \max_{m - \tau \leq k < m}\left\{\frac{\begin{pmatrix}
					k\\m-\tau
			\end{pmatrix}}{\begin{pmatrix}
			m\\m-\tau
		\end{pmatrix}\cdot n^{k - m + \tau}}\right\}
		\end{equation*}
		where $\tau = \vert \mathcal{C}\vert$.
	\end{frame}

	\section{Results and Improvements}
	\begin{frame}{Picnic 2 and Picnic 3}
		By applying Fiat-Shamir paradigm, the $3$-round ZKPoK becomes a NIZKPoK.\pause
		
		By using LowMC as a circuit, the NIZKPoK, in the form of signature, is Picnic 2.\pause
		
		\cite{KalesZ20} then optimized to achieve smaller proof size (and hence signature size) which later became Picnic 3.
	\end{frame}
	
	\begin{frame}{}
		\begin{center}
			{\fontsize{20}{20}\selectfont Thank You!}
		\end{center}
	\end{frame}

	\begin{frame}[allowframebreaks]{References}
		\setbeamertemplate{bibliography item}[text]
		\bibliographystyle{alpha}
		\bibliography{refs.bib}
		
	\end{frame}
	
\end{document}


